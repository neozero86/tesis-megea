%---------------------------------------------------------------------
%
%                      prefacio.tex
%
%---------------------------------------------------------------------
%
% Contiene la p�gina del prefacio.
%
% Se crea como un cap�tulo sin numeraci�n.
%
%---------------------------------------------------------------------

\chapter{Prefacio}

\cabeceraEspecial{Prefacio}

La Ingenier�a es una ciencia aplicada que permite resolver problemas, mediante la aplicaci�n del conocimiento cient�fico. En el caso de Ingenier�a en Inform�tica, el uso del poder de c�lculo de la computadora, en conjunto con los fundamentos de la inform�tica y la ingenier�a del software, posibilitan proveer soluciones integrales de c�mputo, capaces de procesar informaci�n de manera autom�tica.\\

La miner�a de datos es un campo de las ciencias de la computaci�n, el cual permite descubrir patrones en grandes vol�menes de datos. El proceso de miner�a consiste en trabajar sobre los datos, encontrar las caracter�sticas que los relacionan entre s� y utilizar \gls{MAA}, para poder clasificarlos y extraer informaci�n de los mismos.\\

En esta tesis, se trabaja sobre datos neuronales. Se realiza el proceso de miner�a, analizando y evaluando dichos datos desde diferentes enfoques. En el transcurso del mismo se utilizan distintos m�todos estad�sticos de aprendizaje autom�tico, entren�ndolos para clasificar en este particular dominio. El principal objetivo del trabajo es extraer informaci�n del set en cuesti�n y as� contribuir al campo de la neurociencia.


\endinput
