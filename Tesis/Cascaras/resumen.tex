%---------------------------------------------------------------------
%
%                      resumen.tex
%
%---------------------------------------------------------------------

\chapter{Resumen}
\cabeceraEspecial{Resumen}

El sistema nervioso de los mam�feros est� constituido por neuronas que se comunican emitiendo potenciales de acci�n. Cuanta informaci�n es posible rescatar de la actividad de una poblaci�n de neuronas es un problema que a�n no est� definitivamente resuelto. Se sabe que las distintas �reas del cerebro codifican la informaci�n de maneras diferentes, con mayor o menor grado de redundancia e inmunidad a las perturbaciones y con diferentes topolog�as, lo que le confiere a cada �rea propiedades emergentes �nicas.\\

El �rea Tegmental Ventral (VTA) y la Corteza PreFrontal (PFC) son �reas claves para la toma de decisiones y el aprendizaje de conductas que conducen a la obtenci�n de refuerzos apetitivos (comida). Mientras que la primera procesa informaci�n concerniente a la obtenci�n y la predicci�n de refuerzos, la segunda est� involucrada en el procesamiento de reglas. Dada la masiva inervaci�n dopamin�rgica de la primera sobre la segunda, este par de �reas cuenta con la informaci�n necesaria, tanto para aprender reglas de conducta que conllevan al refuerzo, como para eliminar aquellas que resultan superfluas para la tarea.\\

En este trabajo se estudi� cu�nta informaci�n acarrean los potenciales de acci�n \textit{(spikes)} provenientes de las neuronas del �rea Tegmental Ventral (VTA) y de la Corteza Pre-Frontal (PFC) del cerebro de la rata, cuando el animal ejecuta una tarea de discriminaci�n auditiva. Se utilizaron diferentes \gls{maa} para predecir, en base a los potenciales de acci�n emitidos, cual fu� el est�mulo auditivo previamente presentado. Se compar� el rendimiento de dichos algoritmos tomando en cuenta la cantidad de neuronas utilizadas para la predicci�n, la longitud de la ventana de tiempo en la que se observa la actividad y el momento de la observaci�n.

Los resultados encontrados ayudan a definir el esquema de aprendizaje y clasificaci�n m�s adecuado, cuando la decodificaci�n de la informaci�n neuronal debe ser realizada de manera on-line y en electr�nica port�til, con el objetivo de minimizar la invasividad del m�todo.

\endinput
