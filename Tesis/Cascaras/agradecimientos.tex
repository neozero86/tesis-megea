%---------------------------------------------------------------------
%
%                      agradecimientos.tex
%
%---------------------------------------------------------------------
%
% agradecimientos.tex
% Copyright 2009 Marco Antonio Gomez-Martin, Pedro Pablo Gomez-Martin
%
% This file belongs to the TeXiS manual, a LaTeX template for writting
% Thesis and other documents. The complete last TeXiS package can
% be obtained from http://gaia.fdi.ucm.es/projects/texis/
%
% Although the TeXiS template itself is distributed under the 
% conditions of the LaTeX Project Public License
% (http://www.latex-project.org/lppl.txt), the manual content
% uses the CC-BY-SA license that stays that you are free:
%
%    - to share & to copy, distribute and transmit the work
%    - to remix and to adapt the work
%
% under the following conditions:
%
%    - Attribution: you must attribute the work in the manner
%      specified by the author or licensor (but not in any way that
%      suggests that they endorse you or your use of the work).
%    - Share Alike: if you alter, transform, or build upon this
%      work, you may distribute the resulting work only under the
%      same, similar or a compatible license.
%
% The complete license is available in
% http://creativecommons.org/licenses/by-sa/3.0/legalcode
%
%---------------------------------------------------------------------
%
% Contiene la p�gina de agradecimientos.
%
% Se crea como un cap�tulo sin numeraci�n.
%
%---------------------------------------------------------------------

\chapter{Agradecimientos}

\cabeceraEspecial{Agradecimientos}

En primer lugar deseo agradecer al Lic. Camilo Mininni, al Dr. Silvano Zanutto y al Instituto de Ingenier�a Biom�dica de la Facultad de Ingenier�a de la Universidad de Buenos Aires por facilitar los datos de los registros electrofisiol�gicos, sobre los cuales pude realizar esta Tesis.

Considero necesario agradecer a la Universidad de Buenos Aires y en particular a la Facultad de Ingenier�a, por brindarme la posibilidad de tener una educaci�n gratuita, de calidad y de nivel mundial.

Por otro lado, quiero agradecerle a mi tutor, el Dr. Ing. Sergio Lew, quien supo guiarme durante todo el desarrollo de este trabajo. Me gustar�a agradecerle especialmente por la disponibilidad, la velocidad en las respuestas, las correcciones, la dedicaci�n y el tiempo invertido. 

Agradezco a mis padres, Ana y Daniel, por la educaci�n que me brindaron, por estar siempre que los necesit� durante todo el transcurso de mi carrera y por brindarme una base s�lida de conocimientos, que me permiti� aprovechar y disfrutar de la instrucci�n universitaria.

Por �ltimo me gustar�a agradecer a mi mujer, Elizabeth, quien me acompa�� y apoy� incondicionalmente, d�a y noche, e hizo posible la finalizaci�n de esta Tesis de grado.

\endinput
% Variable local para emacs, para  que encuentre el fichero maestro de
% compilaci�n y funcionen mejor algunas teclas r�pidas de AucTeX
%%%
%%% Local Variables:
%%% mode: latex
%%% TeX-master: "../Tesis.tex"
%%% End:
