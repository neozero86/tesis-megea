%-------------------------------------------------------------------
\section{Predicci�n del comportamiento}
\label{capResultados:sec:prediccionComportamiento}

En un experimento futuro, en el momento de registrar los \glspl{S} de las ratas, se podr�a predecir si sacar�n o no la lengua, incluso antes de la reacci�n del propio animal. \\
Con este prop�sito, se agruparon los trials GOc y NOGOi en una nueva clase llamada L (saca lengua). A su vez, los trials GOi y NOGOc fueron agrupados en una nueva clase llamada NO-L (no saca lengua). El an�lisis se realiz� sobre el �rea VTA ya que esta �rea est� relacionada con la predicci�n del comportamiento. \\ %Falta referencia!!
Se prepararon los datos tal y como se especific� en la secci�n~\ref{capSetDeDatos:sec:features:subsec:poblacional:subsubsec:preparacion}, con la salvedad que a la matriz de entrada de los m�todos se le agregaron cinco \textit{trials} NOGOi y cinco GOi, indicando que pertenecen a las clases L y NO-L respectivamente.\\

En la figura~\ref{capResultados:fig:mixedTongueBar} se muestra la performance para la ventana de amplitud $300ms$ que va desde $4000ms$ a $4300ms$. Se tom� est� ventana, debido a que se quiere predecir como responder� el animal al est�mulo antes que se ejecute la respuesta motora.

\figura{Bitmap/10/mixed_tongue_bar}{width=1\linewidth}{capResultados:fig:mixedTongueBar}%
{Gr�fico de la performance en funci�n del tiempo separando las clases L (saca lengua) y NO-L (no saca lengua) para la ventana entre $4000ms$ y $4300ms$}

Para obtener la \gls{Si} de la clasificaci�n, en la tabla~\ref{capResultados:fig:significanciaTongue} se calcul� el valor p, el cual indica de las 500 iteraciones realizadas, cuantas veces la performance de clasificaci�n se encuentra por sobre 0.5.

\begin{equation}
p=1-\frac{\#\ Perf(M_i) > 0.5}{500}
\end{equation}

\figura{Bitmap/10/significancia}{width=1\linewidth}{capResultados:fig:significanciaTongue}%
{Significancia en el que la performance de un m�todo es mayor a 0.5 para la ventana entre $4000ms$ y $4300ms$}



%-------------------------------------------------------------------
\subsection{Conclusi�n}
\label{capResultados:sec:prediccionComportamiento:subsec:conclusion}

La performance de clasificaci�n para cualquiera de los 3 mejores m�todos es aproximadamente $0.7 \pm 0.1$. La \gls{Si} para los mismos es aceptable, demostrando que se alejan del azar.\\
Si bien la precisi�n obtenida no es demasiada, hay que tener en cuenta que se est� prediciendo la respuesta antes que ocurra con un 70\% de probabilidad de acierto.\\
Ademas se pudo observar que para el caso de los primeros milisegundos, con todas las respuestas posibles, es factible utilizar Bayes Naive.