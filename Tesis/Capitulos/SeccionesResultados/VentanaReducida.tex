%-------------------------------------------------------------------
\section[Impacto al variar la longitud de ventana]{Impacto de la longitud de la ventana de an�lisis sobre la performance de clasificaci�n}
\label{capResultados:sec:tamVen}

Dada la baja frecuencia de disparo que tienen en promedio las neuronas de VTA y PFC ($10 Hz$ ~\citep{puryear2010conjunctive} y $5.23 Hz$ ~\citep{baeg2001fast} respectivamente), cabe esperar que el an�lisis sobre ventanas temporales largas provea resultados mas estables que con ventanas cortas, ya que podr�n atraparse mas \glspl{S} en la misma ventana. Sin embargo, poder decidir r�pidamente en funci�n de los est�mulos presentados que respuesta ejecutar es fundamental para la implementaci�n de Interfaces Cerebro Computadora. Aparece entonces, un compromiso entre performance en la decodificaci�n y velocidad de respuesta que debe optimizarse. En esta secci�n se estudiar� el impacto que tiene variar el tama�o de ventana sobre la performance del m�todo Random Forest, analizando cual es la menor amplitud necesaria para abarcar una cantidad suficiente de \glspl{S}, que permitan una clasificaci�n aceptable. Se eligi� este m�todo dado que es el que mejor funciona entre los estudiados en secciones anteriores y con el objetivo de tener una cota superior de performance en la decodificaci�n.\\

Dado que el est�mulo es presentado en el milisegundo $4000$ de la escala temporal, se ampli� paulatinamente el tama�o de ventana desde ese instante (comenzando con una de 100 ms) y se busc� cu�l es el m�nimo tama�o que resulta en una performance significativamente diferente del azar. La figura~\ref{capResultados:fig:windowChange4000RF} muestra el resultado de este an�lisis, se puede observar que con un tama�o de ventana aproximado de 400 ms se logra una performace promedio de 80\%.\\

\figura{Bitmap/09/window_change_from_4000_RF}{width=1\linewidth}{capResultados:fig:windowChange4000RF}%
{Random Forest - Gr�fico de la performance en funci�n al tama�o de ventana, desde el milisegundo $4000$ considerando $500$ corridas}

Surge entonces la pregunta sobre cuando es equivalente utilizar una ventana larga o una ventana deslizante corta. El primer caso asegura obtener toda la informaci�n existente a expensas de un mayor costo computacional, con el consecuente incremento en el consumo energ�tico.\\ La figura~\ref{capResultados:fig:windowVersusRF} muestra la performance promedio del an�lisis con ventanas incrementales y con una ventana de 300 ms deslizante.

\figura{Bitmap/09/window_vs_RF}{width=1\linewidth}{capResultados:fig:windowVersusRF}%
{Random Forest - Ventana de tama�o fijo en $300 ms$ (Verde) vs Ventana ampli�ndose a medida que avanza el tiempo (Rojo).}

%-------------------------------------------------------------------
\subsection{Conclusi�n}
\label{capResultados:sec:aplicaciones:tamVen:conclusion}

Se puede observar que no existen diferencias significativas, m�s all� del hecho de que en el primer caso la performance fue $100\% \pm 0$ a partir de un tama�o de ventana de 1500 ms aproximadamente. Tambi�n se puede deducir que es preferible reducir la carga de procesamiento utilizando una ventana corta deslizante, esperando el tiempo necesario para lograr una performance m�nima aceptable.