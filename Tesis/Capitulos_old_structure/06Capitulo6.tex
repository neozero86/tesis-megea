%---------------------------------------------------------------------
%
%                          Cap�tulo 6
%
%---------------------------------------------------------------------

\chapter{Iteraciones y resultados}
\label{capIterResult}
	
\begin{resumen}
En vista de los resultados obtenidos, y con el objetivo de validar que el porcentaje hallado anteriormente no haya sido casualidad, se corrieron 500 iteraciones para cada ventana.
\end{resumen}
	

%-------------------------------------------------------------------
\section{Introducci�n}
%-------------------------------------------------------------------
\label{capIterResult:sec:introduccion}
Para cada ventana se crear� la matriz $X$ y el vector $Y$ y luego se correr� el algoritmo de clasificaci�n 500 veces generando $X_{RUN}$, $Y_{RUN}$, $X_{TEST}$ e $Y_{TEST}$ tomando los datos de forma aleatoria, con el objetivo de promediar y calcular el desv�o estandar de la performace de clasificaci�n.
\pagebreak
%------------------------------------------------------------------
\section{An�lisis con LDA}
\label{capIterResult:sec:analisis con LDA}

De las 500 iteraciones realizadas, se grafic� el promedio y el desv�o est�ndar de la performance de clasificaci�n en funci�n del tiempo.

\figura{Bitmap/06/LDA_line_run}{width=1\linewidth}{capIterResult:fig:LDALineRun}%
{LDA - Performance en funcion del tiempo - Promedio de las 500 corridas}

Dado que tomamos 500 iteraciones y que la perfomance con su desv�o estandar a partir de cierto instante nunca disminuye de un 70\%, podemos afirmar con toda seguridad que la informaci�n del tono se encuentra almacenada en las neuronas.\\
Se observa ademas que, desestimando algunas ventanas donde el desvio estandar cae, a partir del milisegundo $6000$ la performance con su desvio, se mantiene por encima del 80\%. 
\\\\
A continuaci�n, se correr�n el resto de los algoritmos de clasificaci�n para intentar mejorar la performance obtenida.\\
\pagebreak
%------------------------------------------------------------------
\section{An�lisis con Bayes-Naive}
\label{capIterResult:sec:An�lisis con Bayes-Naive}

Se entren� un calsificador Bayes-Naive con una distribuci�n de datos multinomial, dado que esta distribuci�n suele tener buenos resultados para caracter�sticas discretas. \\
La probabilidad a priori de las clases se configuro como ``empiricas'', es decir que la misma ser� igual a la frecuencia relativa de distribucion de cada clase.
\\
Al igual que con LDA y los dem�s algoritmos, se corrieron 500 iteraciones para luego tomar el promedio de la precisi�n y su desv�o estandar. Se muestran a continuaci�n los resultados graficados en funci�n del tiempo.
\figura{Bitmap/06/BN_line_run}{width=1\linewidth}{capIterResult:fig:BNLineRun}%
{Bayes Naive - Performance en funcion del tiempo - Promedio de las 500 corridas}

Se puede apreciar como a partir del milisegundo $5000$ la performance con su desvio estandar nunca decae del 80\%. A su vez, a partir del milisegundo $6000$ se mantiene sobre el 90\%.
\pagebreak

\section{An�lisis con SVM}
\label{capIterResult:sec:An�lisis con SVM}

Se entren� un clasificador SVM con una funcion de kernel lineal.\\ 

%Puede q false el parametro C

Se presentan, a continuaci�n, los resultados de la performance de clasificaci�n en funci�n del tiempo.

\figura{Bitmap/06/SVM_line_run}{width=1\linewidth}{capIterResult:fig:SVMLineRun}%
{SVM - Performance en funcion del tiempo - Promedio de las 500 corridas}

En este caso a partir del milisegundo $5500$ la performance con su desvio estandar se mantiene sobre el 90\%, y a medida que el tiempo transcurre se obtienen valores casi perfectos. 

\pagebreak
%------------------------------------------------------------------
\section{An�lisis con Random Forest}
\label{capIterResult:sec:An�lisis con Random Forest}

Se entren� un clasificador random forest con 300 �rboles.
El mtry que indica el n�mero m�ximo de variables en cada arbol se configur� en 12. El mismo se calcul� como el piso de la ra�z cuadrada de la cantidad de neuronas. Se aconseja calcular el mtry de esta forma por primera vez, y luego si fuera necesario ir variando el valor hasta obtener la mejor performance. % Fijarse si falta setear algun parametro
\\ 
Se muestran, entonces, los resultados obtenidos hallando el promedio y el desv�o estandar de la performance en las 500 iteraciones.\\

\figura{Bitmap/06/RF_line_run}{width=1\linewidth}{capIterResult:fig:RFLineRun}%
{Random Forest - Performance en funcion del tiempo - Promedio de las 500 corridas}

En la figura~\ref{capIterResult:fig:RFLineRun} se observa que este metodo converge rapidamente a 100\% a partir del milisegundo $4000$. Ya desde el $5000$ la performance se mantiene por encima del 90\% y se incrementa llegando al 100\% y manteniendose estable durante casi todo el resto del tiempo.
\\\\
A simple vista parece ser Random Forest el metodo que mejor clasifica, seguido por SVM y Bayes Naive en ese order. En el capitulo siguiente se realizar�n las comparaciones correspondientes entre los algoritmos, con el objetivo de encontrar el mejor clasificador para \textit {spikes}.  