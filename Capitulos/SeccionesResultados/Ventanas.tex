%-------------------------------------------------------------------
\section[Evoluci�n temporal]{Evoluci�n temporal: an�lisis por ventanas}
%-------------------------------------------------------------------
\label{capResultados:sec:enfoqueVentanas}

Como se mostr� en el cap�tulo~\ref{capSetDeDatos}, la informaci�n contenida en la poblaci�n de neuronas ayuda a predecir el est�mulo presentado con mayor performance que el promedio de lo que consigue cada neurona individualmente. Sin embargo, en ese an�lisis exploratorio sobre la actividad de las neuronas de VTA, se consider� la actividad total de cada neurona dentro de la ventana de an�lisis $(-4000ms, 4000ms)$ alrededor del inicio del tono. Para entender en que momento comienzan las neuronas a tener informaci�n sobre el est�mulo presentado, en esta secci�n se realizar� un an�lisis en ventanas peque�as $(300 ms)$ que se deslizan cada $10 ms$ a lo largo de la duraci�n del trial.\\\\
La actividad total en cada ventana por neurona se utiliz� como \textit{feature} para entrenar los clasificadores y, como antes, la salida deseada para cada trial fue el tipo GO correcto o NOGO correcto. En la figura~\ref{capResultados:fig:windowsTime} se muestra un ejemplo de ventana aplicado sobre la matriz de 0 y 1 original.

\figura{Bitmap/05/windows_time}{width=1\linewidth}{capResultados:fig:windowsTime}%
{Divisi�n de la neurona en ventanas}


De esta manera, para cada neurona, en un instante determinado puede generarse el vector $X_w$ que contiene la actividad en la ventana $w$ y tiene asociado un vector $Y_w$ con la informaci�n acerca del tipo de trial. La figura~\ref{capResultados:fig:XYmatrix} muestra un ejemplo de la estructura de datos construida para entrenar los clasificadores.

\figura{Bitmap/05/XYmatrix}{width=0.9\linewidth}{capResultados:fig:XYmatrix}%
{Composici�n de la matriz $X_w$ y el vector $Y_w$}
 

La figura~\ref{capResultados:fig:LDARunTime} muestra el resultado del an�lisis con el algoritmo LDA, entrenando un clasificador por cada ventana, tomando 70 \% de los trials al azar y verificando con el 30 \% restante.

\figura{Bitmap/05/LDA_run_time}{width=1\linewidth}{capResultados:fig:LDARunTime}%
{LDA - Performance en funci�n del tiempo.}

En la figura~\ref{capResultados:fig:LDARunTime} se puede apreciar claramente como a partir del inicio del tono (4000 ms) la performance comienza a incrementarse. Puede observarse que la performance del clasificador es cercana a lo esperado por azar cuando el tono no fue todav�a presentado.Tambi�n puede notarse la gran fluctuaci�n que hay entre ventanas contiguas, esto se debe principalmente a la variabilidad en los datos. Para determinar si el incremento en la performance es significativo, se realiz� un an�lisis con re-muestreo con el fin de determinar la variabilidad alrededor del valor promedio\\


